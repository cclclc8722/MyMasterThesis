\chapter{緒論}
\label{c:intro}

\section{研究背景}
根據國際數據資訊公司(IDC)表示,在2024年前每年所產生、捕捉、複製、消耗的資料總量將會超過 149 ZB,且多數是非結構化資料,若能具備分析非結構化資料的能力,將能為公司帶來極高的價值。Seagate 預估,到了2025年我們所接觸到的資料中有八成將會是非結構化資料,非結構化資料不僅數量龐大且具有極大的學術發展及商業價值。

根據 IBM 定義,結構化數據(Structured Data)為高度組織化並且易透過機器學習等演算法進行處理的一種資料型態。其優點為不需深入了解不同類型之數據及運作方式,即可做進一步的處理與解釋。然而其缺點為受限之靈活與可用性,結構化資料需儲存於固定格式之數據儲存系統中(如: SQL),若數據需求有所變化時,則必須更新所有數據進而導致大量時間與資源的消耗。而非結構化數據 (Unstructured Data) (包括文字、聲音、圖片、影像等)則無法透過傳統的數據分析方法進行分析。非結構化資料之優勢為可保持未
定義之原始型態,具較高適應性且不需預先定義數據,因此可以更快速且更輕鬆的收集數據而擴大數據池,使資料科學家進行更全面與深入的分析。然而其缺點則為,由於其未定義與非格式化之性質,處理非結構化數據時需仰賴專業知識,如何有效處理非結構化數據成為研究中重要的領域。

而文字訊息對資產報酬的影響早已是財務領域中重要的研究議題。過往研究學者發現市場價格與交易量會受投資者解讀訊息之差異而受影響,認為市場是由理性投資者及非理性投資者組成,而非理性投資者於市場上蒐集訊息,以符合自身對於該資產之預期信念,不同投資者對於訊息的解讀想法將導致不同的決策行為,進而影響相對應的資產價格與市場活動。隨著訊息傳遞研究之發展,實證分析中發現影響投資者決策之訊息中以媒體內容最具影響性,解讀媒體內容後之情緒反應可視為影響資產價格之重要變數,媒體內容、公司之知名度可增加投資者投資該公司的想法 (Merton (1987)、De Long, Shleifer, Summers, and Waldmann (1990a)、Mullainathan and Shleifer (2005)、Duffie (2010))。另從投資者接受媒體內容之特性與理解度進行分析,結果顯示投資者常常只關注廣泛一般訊息,而忽略需要花時間人為處理的詳細訊息,也較易混淆媒體之陳舊程度。因此取得先行消息之投資者可利用消息不對稱的優勢先行決策,使得投資者對於內容之反應具有偏差,因此市場常發生反應落後或過度反應等不效率現象。

整理過去之模型設計,總共可以分成三個類型:分別為建構字典、機器學習法與計量統計模型。建構字典之方法始於傳統語言學,以人工方式編輯成字典;機器學習法與統計計量模型則以 “Learning From Data” 為目標,透過資料處理與模型設計,對資料做進一步的分析。

現今,自然語言處理在許多不同的領域已有很大的發展,然而文本分析應用於財務領域之實證分析仍有很大的進步空間。在財務文字探勘領域,研究文本資料之「情緒」並利用各種不同的方法將其量化為情緒分數(Sentiment Score)為主要研究方向,情感分數可以透過特定方法量化數據後估算而得(e.g. Harvard-IV 社會心理學詞典),接著建構統計模型,並可用於研究財務消息面於金融市場之傳遞情形與不同消息接收度所帶來之影響。

\newpage

\section{研究動機與目的}
最初財務文本使用建構情緒字典方式量化文字情緒,而其缺點為需高度的仰賴人工主觀選字,難以消化隨著時代不斷增加之非結構化資料,也無法因應不斷產生之新詞彙,在預測時無法時時刻刻精確。

有學者提出使用機器學習方法量化文字情緒,然而機器學習模型設計複雜且解釋性不及建構字典法與統計計量方法,因此亦有學著嘗試建構統計計量模型以設計文字情緒量化模型。

近幾年高維度資料問題興起,當解釋變數數量超過所觀測到之樣本數時,標準的迴歸模型(OLS)存在計算上與理論上估計之困難,Ing and Lai (2011) 改良最小平方逐步迴歸法提出 Orthogonal Greedy Algorithm (OGA)解決此類問題,然而該研究尚未應用於具低訊雜比(Signal-to-Noise Ratio)特性之財務文字資料研究上,而與舊有之統計模型相比較(如:SESTM、FARM Predict),執行OGA時電腦所需之運算時間大幅下降,也有足夠的解釋性。本研究預計使用OGA模型於文字情緒量化上之研究並創新應用於文本分析之 OGA Predict 模型;上述之選模方法是建立在當應變數為線性假設之連續報酬的情況下,然而許多文獻支持以二分類報酬(令漲為1、跌或平盤為0)做為文字情緒量化研究之應變數(Fan, et al. (2020), Fan, et al. (2021), Jiang, et al. (2020)),因此本研究也預計使用 Chen, et al.(2019) 改良 CGA 之延伸方法 Chebyshev Greedy Algorithm (CGA) 針對二分類報酬進行建模,並推廣至 CGA Predict 財務文本模型,並基於兩種方法探討在應變數報酬為線性假設或非線性假設的情況下有何異同,並驗證台灣股票市場中之散戶於市場中之反應與績效表現。