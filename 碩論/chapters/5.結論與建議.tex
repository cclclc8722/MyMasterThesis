\chapter{結論與建議}
\label{c:experiment}

\section{結論與建議}
本研究主要依據 Ing and Lai (2011) 及 Chen, Dai, Ing, Lai (2019)所改良之高維度選模模型之文字探勘方法建構台股新聞情緒分數模型,本研究利用模型估算新聞情緒分數而建構交易策略,並評估當應變數為線性假設及非線性假設下投資組合報酬之優劣及新聞反應速度之差異,以驗證散戶為主之台股市場之投組表現。

進行實證分析後發現,透過兩種不同的高維度選模模型所估算的新聞情緒分數確實能有效地預測台股報酬,且當應變數為非線性假設之報酬時能得到更好的投資組合報酬。該情緒分數能精確地捕捉新聞情緒,也能幫助投資者更好的理解新聞媒體是如何影響投資人的交易決策的。在實證分析中我們透過新聞與價格之延遲關係發現,與美國相比,臺灣股票市場在新聞訊息傳遞的速度上有明顯差異,此結果也符合Ke et al. (2019)所提出之結論。除此之外,透過新聞反應速度的分析發現,臺灣的投資人(散戶為主的市場)結構與美國(法人為主的市場)確實有很大差異,此差異為在進行投資決策時,臺灣投資人多有過度悲觀與恐慌反應的情形發生,導致與看見正面新聞而買進股票相比,臺灣投資人更傾向於當負面新聞出現時將股票賣出,此現象也反應在投資組合的報酬之中,也就是與多頭策略相比,空頭策略能夠得到較好的報酬。

本研究最主要目的為驗證當應變數報酬為線性假設或者是非線性假設時,何者可以得到更好的超額報酬並驗證台灣股票市場中之散戶於市場中之反應與績效表現。根據實證結果發現,利用本研究之模型所估算之情緒分數作為投資組合建構依據,其績效優於大盤,也顯示了新聞媒體對於臺灣股票市場不具效率性。對於未來研究方向,除了希望能使用更長期的新聞資料進行研究分析外,也希望能夠解決當新聞文章標註多家公司的問題。